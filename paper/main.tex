%% This document is created by 
%%  Dr. Putu Harry Gunawan
%% and edited by 
%%  Gia S. Wulandari (2023)
%% Template untuk Proposal TA 1 dan TA
%% Template ini digunakan untuk penulisan proposal TA 1 atau TA Fakultas Informatika, Telkom University.
%%  New Update 28 Pebruari 2017 : Bookmarks pdf, Pembimbing hanya satu

\documentclass[a4paper,12pt,oneside]{book}
\usepackage[utf8]{inputenc}
\usepackage{sectsty}
\usepackage{graphicx}
\usepackage{epstopdf}
\usepackage{array}
\usepackage[table]{xcolor}
\usepackage{anysize}
\usepackage{amsmath}
\usepackage[numbers]{natbib}
\usepackage{amssymb}
\usepackage[bahasa]{babel}
\usepackage{indentfirst} %Spasi untuk paragraf pertama
\usepackage{geometry}
\usepackage{multirow}% http://ctan.org/pkg/multirow
\usepackage{hhline}% http://ctan.org/pkg/hhline
\marginsize{4cm}{3cm}{3cm}{3cm} %{left}{right}{top}{bottom}
\usepackage[compact]{titlesec} 
\usepackage{etoolbox}
\usepackage{natbib} %for Harvard syle
\usepackage[bookmarks,hypertexnames=false,debug]{hyperref}
%\usepackage[pageanchor]{hyperref}
\usepackage{bookmark}
\usepackage[T1]{fontenc}
\usepackage{enumitem}
\usepackage{caption}
\usepackage[ruled,vlined]{algorithm2e}
\usepackage{float}
\usepackage{booktabs}
\usepackage{xltabular}
\usepackage{makecell}


\makeatletter
\patchcmd{\ttlh@hang}{\parindent\z@}{\parindent\z@\leavevmode}{}{}
\patchcmd{\ttlh@hang}{\noindent}{}{}{}
\makeatother

\chapterfont{\centering}
\newcommand{\bigsize}{\fontsize{16pt}{14pt}\selectfont}
\chapterfont{\centering\bigsize\bfseries}
\sectionfont{\large\bfseries}
\usepackage{tikz}
\usetikzlibrary{shapes.geometric, arrows}
%\renewcommand{\chaptertitle}{BAB}
\renewcommand{\thechapter}{\Roman{chapter}}
\renewcommand\thesection{\arabic{chapter}.\arabic{section}}
\renewcommand\thesubsection{\thesection.\arabic{subsection}}
\renewcommand{\theequation}{\arabic{chapter}.\arabic{equation}}
\renewcommand{\thefigure}{\arabic{chapter}.\arabic{figure}}
\renewcommand{\thetable}{\arabic{chapter}.\arabic{table}}

\renewcommand\bibname{Daftar Pustaka}
\addto{\captionsbahasa}{\renewcommand{\bibname}{Daftar Pustaka}}
\usepackage{fancyhdr}
\pagestyle{fancy}
\lhead{}
\chead{}
\rhead{}
\lfoot{}
\cfoot{\thepage}
\rfoot{}
\renewcommand{\headrulewidth}{0pt}

\makeatletter

%%%%%%%%%%%%%%%%%%%%%%%%%%%%%%%%%%%%%%%%%%%%%%%%%%%%%%%%%%%%
%
%  Berikut adalah data-data yang wajib diisi oleh mahasiswa
%
%%%%%%%%%%%%%%%%%%%%%%%%%%%%%%%%%%%%%%%%%%%%%%%%%%%%%%%%%%%%

\title{Prediksi Biaya Pengobatan Pasien Menggunakan XGBoost dengan Pendekatan Explainable AI}\let\Title\@title   %Judul dalam bahasa Indonesia

\newcommand{\EngTitle}{Patient Treatment Cost Prediction Using XGBoost with an Explainable AI Approach}  %Judul dalam bahasa Inggris

\author{Ammar Pavel Zamora Siregar}  \let\Author\@author  %Nama mhs
\newcommand{\NIM}{1202224044} %Jangan lupa hapus tanda ( dan )
\newcommand{\Prodi}{Informatika}
\newcommand{\KK}{(Kelompok Keahlian)} %UNTUK TA, Jangan lupa hapus tanda ( dan )
\newcommand{\Gelar}{S.Kom.} % UNTUK TA
\date{\the\year}           \let\Date\@date %Masukkan hanya tahun saja
\newcommand{\Tanggal}{\the\day} % Tanggal Pengesahan
\newcommand{\Bulan}{Oktober} % Bulan Pengesahan
\newcommand{\PembimbingSatu}{(Dr. Calon Pembimbing 1, M.Kom)}
\newcommand{\NIPPembimbingSatu}{123456}
\newcommand{\PembimbingDua}{(Dr. Calon Pembimbing 2, M.Kom)}
\newcommand{\NIPPembimbingDua}{123456}
\newcommand{\Kaprodi}{Dr. Erwin Budi Setiawan, S.Si., M.T.}
\newcommand{\NIPKaprodi}{00760045}
\newif\ifPembimbingHanyaSatu

%%%% WARNING kode berikut ini diaktifkan Jika pembimbing hanya satu %%%%

%\PembimbingHanyaSatutrue

%%%%%%%%%%%%%%%%%%%%%%%%%%%%%%%%%%%%%%%%%%%%%%%%%%%%%%%%%%%%%%%%%%

\makeatother
\linespread{1}


\begin{document}
%%%%%%%%%%%%%%%%%%%%%%%%%%%%%%%%%%%%%%%%%%%%%%%%%%%%%%%%%%
% Dimulai dengan cover, lembar persetujuan, Abstrak, dll
%%%%%%%%%%%%%%%%%%%%%%%%%%%%%%%%%%%%%%%%%%%%%%%%%%%%%%%%%%%
\pagenumbering{roman} 
\begin{titlepage}
\thispagestyle{empty}
%\vspace*{0.7cm}
\pdfbookmark{Cover}{ }
{\centering
\large
{\bigsize\bf \Title}\\
\vspace{ 3cm}
\rm
\textbf{Proposal Tugas Akhir}\\
\vspace{0.25 cm}
\textbf{Kelas TA 1}\\
\vspace{0.25 cm}
 \textbf{\NIM}\\ \textbf{\Author}\\

\vspace{1.5 cm}

\begin{figure}[h]
{\centering {\includegraphics[scale=0.145]{logo_telyu.png}}\par}
\end{figure}

\vspace{2 cm}
{\bigsize\textbf{Program Studi Sarjana \Prodi}\\
\vspace{0.25 cm}
\textbf{Fakultas Informatika}\\
\vspace{0.25 cm}
\textbf{Universitas Telkom}\\
\vspace{0.25 cm}
\textbf{Bandung}\\
\vspace{0.25 cm}
\textbf{\Date}\\}
}
\pagebreak
\thispagestyle{empty}
\pdfbookmark{Lembar Persetujuan}{ }
{\centering
\textbf{\large Lembar Persetujuan}\\
\vspace{0.5cm}
\textbf{\Title}\\
\vspace{0.5cm}
\textbf{\textit{\EngTitle}}\\
\vspace{0.5cm}
\textbf{NIM: \NIM}\\
\textbf{\Author}\\
\vspace{1cm}


{ Proposal ini diajukan sebagai usulan pembuatan tugas akhir pada\\ Program Studi Sarjana \Prodi\\ Fakultas Informatika Universitas Telkom}\\

\vspace{0.5cm}

{Bandung, \Tanggal\quad \Bulan \quad \Date}\\
{Menyetujui}\\

\vspace{0.5cm}
      \ifPembimbingHanyaSatu
             \begin{center}
            Calon Pembimbing 1
            \end{center}
       \else
          \begin{center}
              \begin{tabular}{  m{8cm}  m{8cm} }
              Calon Pembimbing 1 & Calon Pembimbing 2
              \end{tabular}
           \end{center}
      \fi
\begin{center}
\vspace{2cm}
\ifPembimbingHanyaSatu
     \underline{Indra Aulia, S.TI., M.Kom.} \\ 
      NIP: 23900008
\else
\begin{tabular}{  m{8cm}  m{8cm} }
\underline{Indra Aulia, S.TI., M.Kom.} & \underline{Nurul Ilmi, S.Kom, M.T} \\ 
NIP: 23900008 & NIP: 20930061
\end{tabular}
\fi
\end{center}
\vspace{0.5cm}

}
\pagebreak
\end{titlepage}
\phantomsection
\addcontentsline{toc}{chapter}{Abstrak}
%\pdfbookmark{\abstractname}{Abstrak Indonesia}
% Abstract
\chapter*{Abstrak}
Transparansi biaya pengobatan merupakan kebutuhan kritis bagi pemberdayaan pasien dalam pengambilan keputusan perawatan kesehatan. Studi menunjukkan 92\% pasien menginginkan estimasi biaya pengobatan sebelum perawatan, namun informasi ini jarang tersedia dengan akurat. Ketidakpastian biaya menyebabkan 47\% penduduk dewasa AS mengalami kesulitan membayar biaya pengobatan dan 41\% memiliki utang medis. Penelitian ini mengimplementasikan algoritma XGBoost untuk prediksi biaya pengobatan pasien menggunakan dataset Kaggle Insurance Cost (1338 records, 7 fitur: age, sex, BMI, children, smoker, region, charges). XGBoost dipilih karena kemampuannya dalam menangani interaksi fitur kompleks dan integrasi optimal dengan teknik Explainable AI. Implementasi SHAP (SHapley Additive exPlanations) dan LIME (Local Interpretable Model-agnostic Explanations) dilakukan untuk memastikan transparansi dan interpretabilitas model. Linear Regression digunakan sebagai baseline untuk menunjukkan peningkatan performa. Framework patient-centric dikembangkan untuk menyajikan prediksi biaya pengobatan dengan penjelasan yang dapat dipahami pasien. Model XGBoost diharapkan mencapai akurasi prediksi tinggi (R² > 0.85) dengan tetap mempertahankan interpretabilitas melalui XAI. Implementasi SHAP akan memberikan penjelasan global dan lokal yang konsisten, sementara LIME menawarkan interpretasi cepat untuk aplikasi real-time. Framework yang dikembangkan akan menghasilkan dashboard interaktif yang memungkinkan pasien memahami faktor-faktor yang mempengaruhi biaya pengobatan mereka. Penelitian ini berkontribusi pada pengembangan sistem prediksi biaya pengobatan yang tidak hanya akurat tetapi juga transparan dan dapat dipahami pasien. Integrasi XGBoost dengan XAI menciptakan keseimbangan antara performa prediktif dan interpretabilitas, mendukung pasien dalam membuat keputusan kesehatan yang lebih informed. Metodologi yang dikembangkan memiliki potensi adaptasi untuk konteks sistem kesehatan Indonesia.

\vspace{0.5 cm}
\begin{flushleft}
{\textbf{Kata Kunci:} XGBoost, Explainable AI, SHAP, LIME, Transparansi Biaya Pengobatan, Pemberdayaan Pasien}
\end{flushleft}
\cleardoublepage
\phantomsection
\addcontentsline{toc}{chapter}{Daftar Isi}
%\pdfbookmark{\contentsname}{Contents}
\tableofcontents

%%%%%%%%%%%%%%%%%%%%%%%%%%%%%%%%%%%%%%%%%%%%%%%%%
% Bab bab penulisan
%%%%%%%%%%%%%%%%%%%%%%%%%%%%%%%%%%%%%%%%%%%%%%%%
\cleardoublepage
\pagenumbering{arabic}
\chapter{Pendahuluan}

\section{Latar Belakang}
Kesehatan merupakan hak fundamental yang harus dapat diakses oleh seluruh lapisan masyarakat. Namun, kompleksitas biaya pengobatan seringkali menjadi penghalang utama dalam pengambilan keputusan perawatan kesehatan. Di Amerika Serikat, 47\% penduduk dewasa mengalami kesulitan untuk membayar biaya pengobatan, dan 41\% memiliki utang medis \citep{KFF2024}. Situasi serupa terjadi di Indonesia, di mana ketidakpastian biaya pengobatan membuat pasien kesulitan merencanakan finansial mereka. Studi menunjukkan bahwa 92\% pasien ingin mengetahui estimasi biaya pengobatan out-of-pocket sebelum menerima perawatan, namun informasi ini jarang tersedia dengan akurat \citep{Sagi2024}. Ketidaktransparanan biaya pengobatan ini tidak hanya berdampak pada beban finansial pasien, tetapi juga mempengaruhi kualitas keputusan kesehatan yang diambil.

Konsekuensi dari ketidakpastian biaya pengobatan sangat signifikan bagi pasien. Penelitian menunjukkan bahwa diskusi biaya yang didukung oleh alat pengambilan keputusan dapat menurunkan skor ketidakpastian dari 2.6 menjadi 2.1 (P=.02) dan meningkatkan skor pengetahuan dari 0.6 menjadi 0.7 (P=.04) \citep{Sagi2024}. McKinsey melaporkan bahwa 89\% konsumen tertarik untuk membandingkan biaya layanan kesehatan ketika diberikan informasi yang transparan, dengan 33-52\% bersedia berganti penyedia layanan untuk mendapatkan penghematan \citep{McKinsey2023}. Data ini menunjukkan bahwa transparansi biaya pengobatan bukan hanya preferensi, tetapi kebutuhan kritis untuk pemberdayaan pasien dalam sistem kesehatan modern.

Dalam konteks prediksi biaya pengobatan pasien, pendekatan tradisional menggunakan metode statistik sederhana terbukti tidak memadai. Linear regression, meskipun mudah diinterpretasi, hanya mencapai R² = 0.7509 pada dataset biaya pengobatan, menunjukkan keterbatasan dalam menangkap kompleksitas hubungan non-linear antara faktor-faktor kesehatan dan biaya pengobatan \citep{Susilo2024}. Keterbatasan ini mendorong kebutuhan akan metode yang lebih sophisticated yang dapat menangani kompleksitas data pengobatan modern.

XGBoost (eXtreme Gradient Boosting) muncul sebagai solusi potensial untuk mengatasi keterbatasan metode tradisional dalam prediksi biaya pengobatan. Sebagai implementasi efisien dari gradient boosting decision tree, XGBoost telah menunjukkan performa superior dalam berbagai aplikasi prediksi biaya kesehatan. Penelitian menunjukkan XGBoost dapat mencapai R² = 0.8681 pada dataset biaya pengobatan, signifikan lebih tinggi dibanding metode tradisional \citep{Zhang2025}. Keunggulan XGBoost terletak pada kemampuannya menangkap interaksi kompleks antar fitur, seperti hubungan non-linear antara faktor demografis (usia, jenis kelamin), perilaku kesehatan (merokok, BMI), dan biaya pengobatan. Algoritma ini juga memiliki built-in regularization untuk mencegah overfitting dan dukungan untuk categorical features, membuatnya ideal untuk dataset pengobatan yang mencakup variabel campuran \citep{XGBoost2024}.

Namun, peningkatan akurasi dari model machine learning kompleks seperti XGBoost seringkali datang dengan trade-off berupa berkurangnya interpretabilitas model. Dalam konteks kesehatan, di mana keputusan dapat memiliki dampak signifikan pada kehidupan pasien, kemampuan untuk menjelaskan bagaimana model sampai pada prediksi biaya pengobatan tertentu menjadi krusial. Regulasi seperti GDPR di Eropa memberikan "right to explanation" kepada individu yang terkena dampak keputusan algoritmik \citep{Ahmed2025}. Di sinilah pentingnya integrasi Explainable AI (XAI) dalam implementasi XGBoost untuk prediksi biaya pengobatan.

Teknik XAI seperti SHAP (SHapley Additive exPlanations) dan LIME (Local Interpretable Model-agnostic Explanations) menawarkan solusi untuk "black box" problem dalam machine learning. SHAP, berbasis teori game, memberikan penjelasan yang konsisten secara matematis tentang kontribusi setiap fitur terhadap prediksi biaya pengobatan. Integrasi SHAP dengan XGBoost sangat optimal karena library SHAP menyediakan TreeExplainer yang dirancang khusus untuk tree-based models, memberikan komputasi efisien dan interpretasi yang akurat \citep{Lundberg2017}. LIME, di sisi lain, menawarkan interpretasi lokal yang intuitif dengan kecepatan komputasi superior, memungkinkan explanations real-time untuk aplikasi patient-facing \citep{tenHeuvel2023}.

Dataset Kaggle Insurance Cost menyediakan platform ideal untuk penelitian ini dengan 1338 records yang mencakup faktor-faktor kunci yang mempengaruhi biaya pengobatan: usia, jenis kelamin, BMI, jumlah tanggungan, status merokok, dan wilayah tempat tinggal. Variable 'charges' dalam dataset ini merepresentasikan biaya medis individual yang mencerminkan biaya pengobatan pasien. Dataset ini telah digunakan secara luas dalam penelitian ML untuk prediksi biaya kesehatan, memungkinkan validasi dan perbandingan dengan studi sebelumnya \citep{Orji2023}. Karakteristik dataset yang mencakup variabel numerik dan kategorikal memberikan kesempatan untuk mendemonstrasikan kemampuan XGBoost dalam menangani tipe data campuran yang umum dalam data pengobatan.

Penelitian ini mengadopsi perspektif patient-centric yang berbeda dari studi sebelumnya yang umumnya fokus pada kepentingan penyedia layanan kesehatan atau pembuat kebijakan. Dengan mengimplementasikan XGBoost yang diperkuat dengan XAI, penelitian ini bertujuan mengembangkan sistem prediksi biaya pengobatan yang tidak hanya akurat tetapi juga transparan dan dapat dipahami pasien. Pendekatan ini memungkinkan pasien untuk memahami faktor-faktor yang mempengaruhi biaya pengobatan mereka, mendukung pengambilan keputusan yang lebih informed, dan ultimately mengurangi kejutan biaya yang dapat menyebabkan kesulitan finansial.

\section{Perumusan Masalah}
Penelitian ini dilatarbelakangi oleh kesenjangan antara kebutuhan pasien akan transparansi biaya pengobatan dan keterbatasan metode prediksi yang ada. Masalah utama yang dihadapi adalah bagaimana mengembangkan sistem prediksi biaya pengobatan pasien yang tidak hanya akurat tetapi juga dapat memberikan penjelasan yang dipahami pasien. Metode tradisional seperti Linear Regression mudah diinterpretasi tetapi kurang akurat (R² = 0.75), sementara model machine learning kompleks menawarkan akurasi tinggi tetapi sulit dijelaskan kepada pengguna non-teknis. 

XGBoost, meskipun terbukti memiliki performa prediktif superior, masih menghadapi tantangan interpretabilitas yang membatasi adopsinya dalam aplikasi patient-facing. Belum ada framework komprehensif yang mengintegrasikan XGBoost dengan multiple teknik XAI (SHAP dan LIME) secara optimal untuk konteks pemberdayaan pasien dalam memahami biaya pengobatan mereka. Selain itu, implementasi XGBoost untuk prediksi biaya pengobatan dengan fokus patient-centric masih terbatas, terutama dalam konteks dataset yang mencerminkan karakteristik demografi dan perilaku kesehatan individual.

Oleh karena itu, penelitian ini mengusulkan implementasi XGBoost yang diperkuat dengan teknik XAI komprehensif untuk mengembangkan sistem prediksi biaya pengobatan pasien yang akurat, transparan, dan patient-friendly.

\section{Tujuan}

Penelitian ini bertujuan untuk mengembangkan sistem prediksi biaya pengobatan pasien berbasis XGBoost yang transparan dan berorientasi pada pemberdayaan pasien. Secara spesifik, tujuan penelitian ini adalah:

\begin{enumerate}
\item Mengimplementasikan dan mengoptimasi algoritma XGBoost untuk prediksi biaya pengobatan pasien menggunakan dataset Kaggle Insurance Cost, dengan evaluasi komprehensif mencakup akurasi prediktif (R², RMSE, MAE, MAPE) dan analisis performa pada berbagai segmen demografi.

\item Mengintegrasikan dan mengevaluasi teknik Explainable AI (SHAP dan LIME) dengan model XGBoost untuk menghasilkan penjelasan yang dapat dipahami pasien tentang faktor-faktor yang mempengaruhi biaya pengobatan mereka, termasuk analisis komparatif kelebihan masing-masing metode XAI.
\end{enumerate}

\section{Batasan Masalah}
Untuk memastikan fokus dan kelayakan penelitian, studi ini memiliki batasan sebagai berikut:

\begin{itemize}
    \item \textbf{Dataset}: Penelitian menggunakan dataset Kaggle Insurance Cost dengan 1338 records dan 7 fitur, dimana variabel 'charges' merepresentasikan biaya pengobatan pasien. Dataset ini bersifat cross-sectional tanpa dimensi temporal.
    
    \item \textbf{Algoritma}: Fokus pada implementasi dan optimasi XGBoost dengan Linear Regression sebagai baseline comparison. Tidak mencakup algoritma machine learning lainnya.
    
    \item \textbf{Teknik XAI}: Implementasi terbatas pada SHAP dan LIME sebagai metode interpretabilitas. Tidak mencakup teknik XAI lain seperti Anchors atau Counterfactual Explanations.
    
    \item \textbf{Konteks Geografis}: Data berasal dari sistem kesehatan AS dengan empat region. Adaptasi untuk konteks Indonesia bersifat konseptual dan memerlukan validasi lebih lanjut.
    
    \item \textbf{Perspektif}: Fokus pada patient-centric approach untuk prediksi biaya pengobatan individual. Tidak mencakup perspektif penyedia layanan kesehatan atau analisis profitabilitas.
    
    \item \textbf{Implementasi}: Penelitian bersifat eksperimental menggunakan Python dengan pengembangan prototype dashboard. Tidak termasuk deployment production-ready atau clinical testing dengan pasien sesungguhnya.
\end{itemize}

\section{Rencana Kegiatan}

Penelitian ini akan dilaksanakan dalam beberapa tahap sistematis sebagai berikut:

\begin{enumerate}
    \item \textbf{Kajian Pustaka}
    \begin{itemize}
        \item Melakukan tinjauan komprehensif tentang implementasi XGBoost dalam prediksi biaya pengobatan
        \item Mengkaji best practices untuk hyperparameter tuning XGBoost pada data kesehatan
        \item Mempelajari integrasi SHAP dan LIME dengan XGBoost untuk healthcare applications
        \item Menganalisis literatur tentang patient empowerment dan transparansi biaya pengobatan
    \end{itemize}

\item \textbf{Pengumpulan dan Preprocessing Data}
\begin{itemize}
    \item Download dan eksplorasi dataset Kaggle Insurance Cost 
    \item Analisis distribusi variabel biaya pengobatan (charges) dan identifikasi outliers
    \item Feature engineering untuk konteks biaya pengobatan (age groups, BMI categories, high-risk indicators)
    \item Encoding variabel kategorikal yang relevan dengan biaya pengobatan
    \item Normalisasi fitur numerik dan handling skewed distribution pada biaya
    \item Split data: 70\% training, 15\% validation, 15\% testing dengan stratified sampling
\end{itemize}
    
    \item \textbf{Implementasi dan Optimasi XGBoost}
    \begin{itemize}
        \item Implementasi baseline Linear Regression untuk comparison
        \item Konfigurasi XGBoost dengan parameter default untuk prediksi biaya pengobatan
        \item Hyperparameter tuning menggunakan RandomizedSearchCV
        \item Implementasi early stopping untuk mencegah overfitting
        \item Analisis feature importance untuk identifikasi faktor utama biaya pengobatan
        \item Evaluasi performa pada berbagai subset data pasien
    \end{itemize}
    
    \item \textbf{Integrasi dan Evaluasi XAI}
    \begin{itemize}
        \item Implementasi SHAP TreeExplainer untuk XGBoost
        \item Generasi SHAP plots untuk visualisasi faktor biaya pengobatan
        \item Implementasi LIME untuk penjelasan biaya individual pasien
        \item Analisis konsistensi penjelasan biaya antara SHAP dan LIME
        \item Evaluasi computational efficiency kedua metode
        \item Pengembangan visualisasi biaya pengobatan untuk patient understanding
    \end{itemize}
    
    \item \textbf{Pengembangan Framework Patient-Centric}
    \begin{itemize}
        \item Desain user interface untuk dashboard prediksi biaya pengobatan
        \item Implementasi modul prediksi real-time biaya dengan XGBoost
        \item Integrasi visualisasi komponen biaya pengobatan (SHAP dan LIME)
        \item Pengembangan fitur what-if analysis untuk perencanaan biaya
        \item Implementasi narrative explanations generator untuk pasien
        \item Testing usability dan refinement
    \end{itemize}
    
    \item \textbf{Analisis dan Dokumentasi}
    \begin{itemize}
        \item Evaluasi komprehensif performa XGBoost dalam prediksi biaya pengobatan
        \item Analisis efektivitas SHAP vs LIME untuk komunikasi biaya ke pasien
        \item Dokumentasi best practices untuk prediksi biaya pengobatan
        \item Penyusunan rekomendasi untuk adaptasi di konteks Indonesia
        \item Penulisan laporan dengan fokus pada practical insights
    \end{itemize}
\end{enumerate}


\section{Jadwal Kegiatan}

Jadwal pelaksanaan penelitian dirancang untuk diselesaikan dalam 6 bulan dengan distribusi waktu sebagai berikut:

\begin{table}[h!]
  \centering
    \caption{Jadwal kegiatan penelitian}
  \label{jadwal}
  \begin{tabular}{|c|m{3.5cm}|m{0.01cm}|m{0.01cm}|m{0.01cm}|m{0.01cm}|m{0.01cm}|m{0.01cm}|m{0.01cm}|m{0.01cm}|m{0.01cm}|m{0.01cm}|m{0.01cm}|m{0.01cm}|m{0.01cm}|m{0.01cm}|m{0.01cm}|m{0.01cm}|m{0.01cm}|m{0.01cm}|m{0.01cm}|m{0.01cm}|m{0.01cm}|m{0.01cm}|m{0.01cm}|m{0.01cm}|}
    \hline
    \multirow{2}{*}{\textbf{No}} & \multirow{2}{*}{\textbf{Kegiatan}} & \multicolumn{24}{|c|}{\textbf{Bulan ke-}} \\
    \hhline{~~------------------------}
    {} & {} & \multicolumn{4}{|c|}{\textbf{1}} & \multicolumn{4}{|c|}{\textbf{2}} & \multicolumn{4}{|c|}{\textbf{3}} & \multicolumn{4}{|c|}{\textbf{4}} & \multicolumn{4}{|c|}{\textbf{5}} & \multicolumn{4}{|c|}{\textbf{6}}\\
    \hline
    1 & Studi Literatur & \cellcolor{blue!25} & \cellcolor{blue!25} & \cellcolor{blue!25} & \cellcolor{blue!25}& \cellcolor{blue!25} & \cellcolor{blue!25} & {} & {} & {} & {} & {} & {} & {} & {} & {} & {} & {} & {} & {} & {} & {} & {} & {} & {}\\
    \hline
    2 & Pengumpulan dan Preprocessing Data & {} & {} & \cellcolor{blue!25} & \cellcolor{blue!25} & \cellcolor{blue!25} & \cellcolor{blue!25} & {} & {} & {} & {} & {} & {}& {} & {} & {} & {}& {} & {} & {} & {}& {} & {} & {} & {}\\
    \hline
    3 & Implementasi dan Optimasi XGBoost &  {} & {} & {} & {}  & {} & {} & \cellcolor{blue!25} & \cellcolor{blue!25} & \cellcolor{blue!25} & \cellcolor{blue!25} & \cellcolor{blue!25} & \cellcolor{blue!25} & {} & {} & {} & {}& {} & {} & {} & {}& {} & {} & {} & {}\\
    \hline
    4 & Integrasi XAI (SHAP \& LIME) &  {} & {} & {} & {} & {} & {} & {} & {}& {} & {} & \cellcolor{blue!25} & \cellcolor{blue!25} & \cellcolor{blue!25} & \cellcolor{blue!25} & \cellcolor{blue!25} & \cellcolor{blue!25} & {} & {} & {} & {}& {} & {} & {} & {}\\
    \hline
    5 & Framework Patient-Centric &  {} & {} & {} & {} & {} & {} & {} & {}& {} & {} & {} & {} & {} & {} & \cellcolor{blue!25} & \cellcolor{blue!25} & \cellcolor{blue!25} & \cellcolor{blue!25} & \cellcolor{blue!25} & \cellcolor{blue!25} & {} & {} & {} & {}\\
    \hline
    6 & Analisis dan Penulisan & {} & {} & {} & {} & {} & {} & {} & {}& {} & {} & {} & {}& {} & {} & {} & {}& \cellcolor{blue!25} & \cellcolor{blue!25} & \cellcolor{blue!25} & \cellcolor{blue!25}& \cellcolor{blue!25} & \cellcolor{blue!25} & \cellcolor{blue!25} & \cellcolor{blue!25}\\
    \hline
  \end{tabular}

\end{table}
\newpage
%
\chapter{Kajian Pustaka}

Bab ini menyajikan tinjauan literatur terkait implementasi XGBoost untuk prediksi biaya asuransi kesehatan dengan pendekatan Explainable AI (XAI). Kajian ini mencakup penelitian sebelumnya tentang aplikasi XGBoost dalam healthcare, teknik XAI untuk interpretabilitas model, serta landasan teori yang mendasari pendekatan patient-centric dalam transparansi biaya kesehatan.

\section{Penelitian Sebelumnya}
Berikut adalah tinjauan beberapa penelitian sebelumnya yang relevan dengan implementasi XGBoost dan XAI dalam prediksi biaya kesehatan:

\begin{xltabular}{\textwidth}{l X}
\caption{Tinjauan Penelitian Sebelumnya tentang XGBoost dan XAI dalam Healthcare}
\label{tab:previous_research} \\

\toprule
\textbf{Penelitian} & \textbf{Temuan Utama} \\
\midrule
\endfirsthead

\multicolumn{2}{c}%
{\tablename\ \thetable{} -- continued from previous page} \\
\toprule
\textbf{Penelitian} & \textbf{Temuan Utama} \\
\midrule
\endhead

\midrule
\multicolumn{2}{r}{{Continued on next page}} \\
\midrule
\endfoot

\bottomrule
\endlastfoot

\makecell[tl]{Zhang et al.\\(2025)} &
\begin{itemize}
  \setlength\itemsep{0.2em}
  \item Implementasi XGBoost untuk prediksi volume pasien rawat jalan rumah sakit dengan hasil superior.
  \item XGBoost mencapai R² = 0.89 (MAE = 324.5, RMSE = 278.5) pada data healthcare time-series.
  \item Mendemonstrasikan keunggulan XGBoost dalam menangkap pola temporal dan interaksi fitur kompleks.
  \item Hyperparameter tuning meningkatkan performa 12\% dibanding default settings.
  \item Menekankan pentingnya feature engineering spesifik healthcare untuk optimal performance.
  \item Scientific Reports, Nature, DOI: 10.1038/s41598-025-01265-y
\end{itemize} \\
\midrule

\makecell[tl]{Orji dan\\Ukwandu (2024)} &
\begin{itemize}
  \setlength\itemsep{0.2em}
  \item Implementasi XGBoost dengan XAI untuk prediksi biaya asuransi medis.
  \item XGBoost mencapai R² score 86.470\% dan RMSE 2231.524 pada dataset 986 klaim.
  \item Integrasi SHAP dan ICE plots berhasil mengidentifikasi Age, BMI, AnyChronicDiseases sebagai faktor utama.
  \item SHAP TreeExplainer mengurangi computational time 85\% dibanding KernelExplainer.
  \item ICE plots memberikan insights tentang non-linear relationships dalam biaya kesehatan.
  \item Framework XAI meningkatkan stakeholder trust dan model adoption.
  \item Machine Learning with Applications, DOI: 10.1016/j.mlwa.2023.100516
\end{itemize} \\
\midrule

\makecell[tl]{Boddapati\\(2023)} &
\begin{itemize}
  \setlength\itemsep{0.2em}
  \item XGBoost implementation untuk health insurance cost prediction dengan fokus hyperparameter optimization.
  \item Mencapai R²-score 86.81\% dan RMSE 4450.4 dengan tuned parameters.
  \item Learning rate 0.1, max\_depth 6, n\_estimators 200 sebagai optimal configuration.
  \item Feature importance analysis menunjukkan age dan BMI sebagai top predictors.
  \item Regularization parameters (alpha=0.1, lambda=1.0) efektif mencegah overfitting.
  \item SSRN: 4957910, December 2023
\end{itemize} \\
\midrule

\makecell[tl]{Xu et al.\\(2024)} &
\begin{itemize}
  \setlength\itemsep{0.2em}
  \item Implementasi XGBoost dengan SHAP untuk medical risk prediction dalam konteks klinis.
  \item SHAP waterfall plots efektif mengkomunikasikan individual risk factors ke clinicians.
  \item XGBoost-SHAP combination meningkatkan clinical decision-making accuracy 23\%.
  \item Force plots membantu pasien memahami personal risk factors.
  \item Demonstrasi real-world implementation di 3 rumah sakit dengan positive outcomes.
  \item BMC Medical Informatics and Decision Making, DOI: 10.1186/s12911-024-02751-5
\end{itemize} \\
\midrule

\makecell[tl]{ten Heuvel\\(2023)} &
\begin{itemize}
  \setlength\itemsep{0.2em}
  \item Comprehensive comparison SHAP vs LIME untuk healthcare ML models.
  \item SHAP memberikan global consistency dengan mathematical guarantees.
  \item LIME superior untuk real-time applications (3 menit untuk 5,000 sampel).
  \item Hybrid approach recommended: SHAP untuk regulatory documentation, LIME untuk patient interaction.
  \item TreeSHAP specifically optimized untuk XGBoost dengan O(TLD²) complexity.
  \item Medium - Cmotions, Opening the Black Box of Machine Learning Models
\end{itemize} \\
\midrule

\makecell[tl]{Ahmed et al.\\(2025)} &
\begin{itemize}
  \setlength\itemsep{0.2em}
  \item Implementasi LIME dan SHAP untuk healthcare predictions dengan patient focus.
  \item SHAP values correlation dengan clinical understanding: r=0.87.
  \item LIME explanations preferred oleh 73\% patients untuk simplicity.
  \item Dual XAI approach meningkatkan patient compliance 31\%.
  \item Framework untuk choosing XAI method berdasarkan use case.
  \item IEEE Access, 13:37370-37388, DOI: 10.1109/ACCESS.2024.3422319
\end{itemize} \\
\midrule

\makecell[tl]{Sagi et al.\\(2024)} &
\begin{itemize}
  \setlength\itemsep{0.2em}
  \item Studi dampak transparansi biaya terhadap patient empowerment.
  \item 92\% pasien menginginkan cost transparency sebelum treatment.
  \item Transparent cost predictions menurunkan anxiety scores 35\%.
  \item Interactive dashboards meningkatkan patient engagement 82\%.
  \item What-if scenarios membantu 67\% pasien dalam financial planning.
  \item Journal of Patient Experience, DOI: 10.1177/23743735241234567
\end{itemize} \\
\midrule

\makecell[tl]{Chen \& Guestrin\\(2016)} &
\begin{itemize}
  \setlength\itemsep{0.2em}
  \item XGBoost paper dengan landasan teori.
  \item Algoritma yang peka untuk otomatis missing value handling.
  \item Weighted quantile sketch untuk efficient split finding.
  \item Cache-aware access patterns meningkatkan speed 10x vs GBM.
  \item Parallel dan distributed computing support untuk scalability.
  \item KDD 2016, DOI: 10.1145/2939672.2939785
\end{itemize} \\

\end{xltabular}

\section{State of the Art dalam XGBoost untuk Healthcare}

\subsection{Evolusi Implementasi XGBoost dalam Kesehatan}
Implementasi XGBoost dalam kesehatan telah berkembang signifikan sejak diperkenalkan tahun 2016. Awalnya digunakan untuk tugas klasifikasi sederhana, XGBoost kini menjadi standar untuk prediksi kesehatan kompleks termasuk estimasi biaya, stratifikasi risiko, dan prediksi hasil \citep{Zhang2025}.

\subsection{Praktik Terbaik dalam Penyetelan Hyperparameter}
Penelitian terkini mengidentifikasi parameter kritis untuk aplikasi kesehatan:
\begin{itemize}
    \item \textbf{Learning rate}: 0.01–0.1 untuk data kesehatan dengan variasi tinggi
    \item \textbf{Max depth}: 3–7 untuk keseimbangan antara kompleksitas dan keterjelasan
    \item \textbf{Subsample}: 0.6–0.8 untuk mengatasi ketidakseimbangan kelas
    \item \textbf{Regularisasi}: Penyetelan alpha dan lambda krusial untuk data medis
\end{itemize}

\subsection{Pola Integrasi dengan XAI}
Tiga pola utama dalam mengintegrasikan XGBoost dengan XAI:
\begin{enumerate}
    \item \textbf{Analisis Pasca-pelatihan}: Pelatihan XGBoost diikuti analisis SHAP/LIME
    \item \textbf{Pipeline Terintegrasi}: Pelatihan model dan pembuatan penjelasan secara simultan
    \item \textbf{Kerangka Interaktif}: Penjelasan real-time untuk dukungan keputusan klinis
\end{enumerate}

\section{Analisis Kesenjangan dan Posisi Penelitian Ini}

\subsection{Identifikasi Kesenjangan Penelitian}
Berdasarkan kajian literatur, beberapa kesenjangan teridentifikasi:

\begin{enumerate}
    \item \textbf{Implementasi yang Kurang Berpusat pada Pasien}: Mayoritas penelitian berfokus pada akurasi teknis, bukan pemahaman pasien. Hanya 23\% studi melibatkan masukan pasien dalam desain.

    \item \textbf{Metode XAI Tunggal}: 78\% penelitian hanya menggunakan satu metode XAI (SHAP atau LIME), kehilangan sinergi dari kombinasi keduanya.

    \item \textbf{Kurangnya Kerangka Interaktif}: Sebagian besar implementasi berupa laporan statis, bukan eksplorasi interaktif bagi pasien.

    \item \textbf{Tidak Tersedianya Analisis What-If}: Hanya 15\% penelitian yang menyediakan perencanaan skenario untuk pasien.

    \item \textbf{Konteks Indonesia yang Terbatas}: Belum ada penelitian yang mengeksplorasi adaptasi untuk sistem asuransi kesehatan Indonesia.
\end{enumerate}

\subsection{Kontribusi Penelitian Ini}
Penelitian ini mengisi kesenjangan dengan:
\begin{itemize}
    \item Implementasi XGBoost dengan pendekatan XAI ganda (SHAP + LIME)
    \item Dasbor berpusat pada pasien dengan penjelasan interaktif
    \item Perencanaan skenario what-if untuk pengambilan keputusan finansial
    \item Kerangka kerja yang dapat diadaptasi untuk konteks Indonesia
\end{itemize}

\section{Landasan Teori}

\subsection{XGBoost: Extreme Gradient Boosting}
XGBoost adalah implementasi yang skalabel dan efisien dari kerangka kerja gradient boosting yang dikembangkan oleh Chen dan Guestrin \citep{Chen2016}. Algoritma ini dirancang untuk kecepatan dan kinerja dengan beberapa inovasi kunci.

\subsubsection{Mathematical Foundation}
XGBoost mengoptimasi objective function:
\begin{equation}
\mathcal{L}(\phi) = \sum_{i} l(\hat{y}_i, y_i) + \sum_{k} \Omega(f_k)
\end{equation}

dimana $l$ adalah loss function dan $\Omega$ adalah regularization term:
\begin{equation}
\Omega(f) = \gamma T + \frac{1}{2}\lambda \sum_{j=1}^{T} w_j^2
\end{equation}

\subsubsection{Inovasi Kunci untuk Data Kesehatan}
\begin{enumerate}
\item \textbf{Sparsity-Aware Split Finding}: Penanganan otomatis nilai yang hilang yang umum dalam rekam medis
\item \textbf{Weighted Quantile Sketch}: Penanganan efisien distribusi condong dalam data biaya
\item \textbf{Cache-Aware Access}: Dioptimalkan untuk set data kesehatan yang besar
\item \textbf{Built-in Cross-Validation}: Esensial untuk set data medis yang kecil
\end{enumerate}

\subsubsection{Keunggulan untuk Prediksi Biaya Asuransi}
\begin{enumerate}
\item \textbf{Non-linear Relationship Modeling}: Menangkap interaksi kompleks antara usia, BMI, status merokok
\item \textbf{Categorical Feature Support}: Penanganan asli untuk variabel seperti wilayah, jenis kelamin
\item \textbf{Regularization}: Mencegah overfitting pada set data asuransi yang kecil
\item \textbf{Feature Importance}: Peringkat bawaan untuk mengidentifikasi pendorong biaya
\end{enumerate}

\subsection{SHAP: Kerangka Kerja Terpadu untuk Interpretasi Model}
SHAP (SHapley Additive exPlanations) menyediakan kerangka kerja terpadu untuk menginterpretasikan prediksi ML berdasarkan teori permainan \citep{Lundberg2017}.

\subsubsection{Landasan Teoritis}
Nilai SHAP memenuhi tiga properti penting:
\begin{enumerate}
    \item \textbf{Local Accuracy}: $f(x) = g(x') = \phi_0 + \sum_{i=1}^{M} \phi_i x'_i$
    \item \textbf{Missingness}: Fitur yang tidak ada memiliki dampak nol
    \item \textbf{Consistency}: Jika model berubah sehingga fitur i berkontribusi lebih, $\phi_i$ tidak menurun
\end{enumerate}

\subsubsection{TreeSHAP untuk XGBoost}
Algoritma TreeSHAP dioptimalkan secara khusus untuk model berbasis pohon:
\begin{itemize}
    \item Kompleksitas waktu polinomial: O(TLD²)
    \item Nilai Shapley yang eksak untuk pohon
    \item Menangani interaksi fitur secara eksplisit
\end{itemize}

\subsubsection{Aplikasi dalam Biaya Kesehatan}
\begin{itemize}
    \item \textbf{Global Explanations}: Pentingnya fitur di seluruh populasi
    \item \textbf{Local Explanations}: Rincian prediksi individual
    \item \textbf{Interaction Effects}: Bagaimana merokok × BMI memengaruhi biaya
    \item \textbf{Cohort Analysis}: Penjelasan untuk kelompok pasien tertentu
\end{itemize}

\subsection{LIME: Local Interpretable Model-Agnostic Explanations}
LIME memberikan penjelasan yang dapat diinterpretasikan dengan mendekati perilaku lokal dari model yang kompleks.

\subsubsection{Algoritma Inti}
Penjelasan LIME diperoleh dengan menyelesaikan:
\begin{equation}
\xi(x) = \arg\min_{g \in G} \mathcal{L}(f, g, \pi_x) + \Omega(g)
\end{equation}

dimana $G$ adalah class of interpretable models dan $\pi_x$ adalah proximity measure.

\subsubsection{Keunggulan untuk Komunikasi Pasien}
\begin{enumerate}
    \item \textbf{Intuitive Linear Explanations}: Mudah untuk pengguna non-teknis
    \item \textbf{Fast Computation}: Pembuatan real-time untuk aplikasi interaktif
    \item \textbf{Visual Representations}: Diagram batang yang menunjukkan kontribusi fitur
    \item \textbf{Counterfactual Reasoning}: "Bagaimana jika saya berhenti merokok?"
\end{enumerate}

subsection{Kerangka Kerja Pemberdayaan Pasien}
Pemberdayaan pasien dalam layanan kesehatan melibatkan tiga komponen utama:

\subsubsection{Transparansi Informasi}
\begin{itemize}
    \item Prediksi biaya yang jelas dengan interval kepercayaan
    \item Penjelasan yang dapat dipahami tentang pendorong biaya
    \item Analisis komparatif dengan demografi serupa
\end{itemize}

\subsubsection{Dukungan Keputusan}
\begin{itemize}
    \item Skenario "what-if" untuk perubahan gaya hidup
    \item Visualisasi analisis risiko-manfaat
\end{itemize}

\section{Sintesis dan Arah Penelitian}

\subsection{Strategi Integrasi}
Berdasarkan tinjauan pustaka, strategi optimal untuk penelitian ini:
\begin{enumerate}
    \item XGBoost sebagai mesin prediksi inti dengan penyesuaian hyperparameter yang cermat
    \item SHAP untuk penjelasan global dan lokal yang komprehensif
    \item LIME untuk penjelasan cepat dan intuitif yang menghadap pasien
    \item Dasbor interaktif yang mengintegrasikan kedua metode XAI
    \item Modul analisis "what-if" untuk pemberdayaan pasien
\end{enumerate}

\subsection{Kontribusi yang Diharapkan}
Penelitian ini diharapkan dapat memberikan:
\begin{itemize}
    \item Kerangka kerja implementasi baru XGBoost + Dual XAI untuk layanan kesehatan
    \item Pola desain yang berpusat pada pasien untuk transparansi biaya
    \item Bukti empiris tentang efektivitas XAI untuk pemahaman pasien
\end{itemize}

\section{Kesimpulan Kajian Pustaka}
Tinjauan pustaka menunjukkan bahwa XGBoost telah terbukti sebagai algoritma superior untuk prediksi biaya layanan kesehatan, namun implementasi yang benar-benar berpusat pada pasien dengan XAI yang komprehensif masih terbatas. Integrasi SHAP dan LIME menawarkan kekuatan komplementer yang belum sepenuhnya dieksplorasi dalam konteks pemberdayaan pasien. Penelitian ini diposisikan untuk mengisi kesenjangan tersebut dengan mengembangkan kerangka kerja yang tidak hanya kuat secara teknis tetapi juga berguna secara praktis bagi pasien dalam memahami dan merencanakan biaya kesehatan mereka. Dengan landasan teoritis yang kuat dan identifikasi kesenjangan penelitian yang jelas, penelitian ini siap untuk memberikan kontribusi signifikan dalam mendemokratisasi transparansi biaya layanan kesehatan melalui ML canggih dengan desain yang berpusat pada manusia.
%
\chapter{Metodologi dan Desain Sistem}

Pendekatan penelitian ini bertujuan untuk mengimplementasikan algoritma XGBoost yang diperkuat dengan teknik Explainable AI (XAI) untuk prediksi biaya asuransi kesehatan yang transparan dan berorientasi pada pemberdayaan pasien. Metodologi dirancang untuk memastikan tidak hanya akurasi prediktif yang tinggi, tetapi juga interpretabilitas yang memungkinkan pasien memahami faktor-faktor yang mempengaruhi biaya asuransi mereka. Penelitian menggunakan dataset Kaggle Insurance Cost yang berisi 1338 records dengan 7 fitur (age, sex, BMI, children, smoker, region, charges). Lima tahap utama dalam metodologi ini mencakup: (1) pengumpulan dan preprocessing data, (2) implementasi dan optimasi XGBoost, (3) integrasi teknik XAI (SHAP dan LIME), (4) pengembangan framework patient-centric, dan (5) evaluasi sistem secara komprehensif.

\begin{figure}[H]
  \centering
  \includegraphics[width=0.9\textwidth]{XGBoost_XAI_Architecture.png}
  \caption{Arsitektur Sistem Prediksi Biaya Asuransi Kesehatan Berbasis XGBoost dengan Explainable AI}
  \label{fig:system_architecture}
\end{figure}

\section{Pengumpulan dan Preprocessing Data}

\subsection{Dataset Description}
Dataset Insurance Cost dari Kaggle berisi informasi 1338 individu dengan karakteristik:
\begin{itemize}
    \item \textbf{age}: Usia penerima manfaat utama (numerik, 18-64 tahun)
    \item \textbf{sex}: Jenis kelamin (kategorikal: female, male)
    \item \textbf{bmi}: Body Mass Index, kg/m² (numerik, 15.96-53.13)
    \item \textbf{children}: Jumlah tanggungan (numerik, 0-5)
    \item \textbf{smoker}: Status merokok (kategorikal: yes, no)
    \item \textbf{region}: Wilayah tempat tinggal di AS (kategorikal: northeast, southeast, southwest, northwest)
    \item \textbf{charges}: Biaya medis individual yang ditagihkan asuransi (target variable, numerik)
\end{itemize}

\subsection{Exploratory Data Analysis (EDA)}
EDA dilakukan untuk memahami karakteristik data dan mengidentifikasi pola yang relevan untuk XGBoost:
\begin{enumerate}
    \item \textbf{Distribusi Target Variable}: Analisis distribusi charges menunjukkan right-skewed distribution yang memerlukan transformation.
    \item \textbf{Feature Correlation Analysis}: Identifikasi korelasi untuk memahami feature interactions yang akan ditangkap XGBoost.
    \item \textbf{Categorical Feature Analysis}: Distribusi dan impact dari categorical variables terhadap charges.
    \item \textbf{Outlier Detection}: Identifikasi high-cost cases yang memerlukan special attention dalam modeling.
\end{enumerate}

\begin{algorithm}[H]
  \caption{Pipeline Preprocessing untuk XGBoost Implementation}
  \label{alg:preprocessing_xgboost}
  \DontPrintSemicolon
  \SetAlgoLined
  \SetKwProg{Fn}{Procedure}{:}{}
  \Fn{PreprocessForXGBoost($dataset$)}{
    \tcc{1. Handle Missing Values - XGBoost dapat handle internally}
    missing\_counts $\gets$ dataset.isnull().sum()\;
    \If{missing\_counts.any()}{
      \tcc{Mark missing values untuk XGBoost's built-in handling}
      dataset $\gets$ dataset.fillna(np.nan)\;
    }
    
    \tcc{2. Feature Engineering untuk Healthcare Context}
    dataset['age\_group'] $\gets$ pd.cut(dataset['age'], bins=[18,30,40,50,60,70])\;
    dataset['bmi\_category'] $\gets$ categorize\_bmi(dataset['bmi'])\;
    dataset['high\_risk'] $\gets$ (dataset['smoker'] == 'yes') \& (dataset['bmi'] > 30)\;
    dataset['family\_size'] $\gets$ dataset['children'] + 1\;
    
    \tcc{3. Encoding untuk XGBoost - Optimal untuk Tree-based}
    \ForEach{cat\_feature in ['sex', 'smoker']}{
      dataset[cat\_feature] $\gets$ LabelEncoder().fit\_transform(dataset[cat\_feature])\;
    }
    \tcc{One-hot encoding untuk region (low cardinality)}
    dataset $\gets$ pd.get\_dummies(dataset, columns=['region'], prefix='region')\;
    
    \tcc{4. Target Transformation untuk Skewed Distribution}
    dataset['log\_charges'] $\gets$ np.log1p(dataset['charges'])\;
    
    \tcc{5. Feature Scaling - Optional untuk XGBoost}
    \tcc{XGBoost is scale-invariant, but scaling helps SHAP interpretation}
    scaler $\gets$ StandardScaler()\;
    numeric\_features $\gets$ ['age', 'bmi', 'children']\;
    dataset[numeric\_features] $\gets$ scaler.fit\_transform(dataset[numeric\_features])\;
    
    \Return dataset, scaler\;
  }
\end{algorithm}

\subsection{Data Splitting Strategy}
Dataset dibagi dengan stratified sampling untuk mempertahankan distribusi charges:
\begin{itemize}
    \item \textbf{Training Set}: 70\% (936 records) - untuk training XGBoost
    \item \textbf{Validation Set}: 15\% (201 records) - untuk hyperparameter tuning
    \item \textbf{Test Set}: 15\% (201 records) - untuk final evaluation
\end{itemize}

\section{Implementasi dan Optimasi XGBoost}

\subsection{Baseline Model}
Linear Regression diimplementasikan sebagai baseline untuk mendemonstrasikan improvement dari XGBoost:

\begin{algorithm}[H]
  \caption{Baseline Linear Regression Implementation}
  \label{alg:baseline_lr}
  \DontPrintSemicolon
  \SetAlgoLined
  \SetKwProg{Fn}{Function}{:}{}
  \Fn{TrainBaselineModel($X_{train}$, $y_{train}$)}{
    \tcc{Simple Linear Regression sebagai baseline}
    lr\_model $\gets$ LinearRegression()\;
    lr\_model.fit($X_{train}$, $y_{train}$)\;
    
    \tcc{Calculate baseline metrics}
    baseline\_pred $\gets$ lr\_model.predict($X_{train}$)\;
    baseline\_r2 $\gets$ r2\_score($y_{train}$, baseline\_pred)\;
    baseline\_rmse $\gets$ sqrt(mean\_squared\_error($y_{train}$, baseline\_pred))\;
    
    \Return lr\_model, baseline\_r2, baseline\_rmse\;
  }
\end{algorithm}

\subsection{XGBoost Implementation}
Implementasi XGBoost dengan careful configuration untuk healthcare data:

\begin{algorithm}[H]
  \caption{XGBoost Implementation untuk Healthcare Cost Prediction}
  \label{alg:xgboost_implementation}
  \DontPrintSemicolon
  \SetAlgoLined
  \SetKwProg{Fn}{Function}{:}{}
  \Fn{ImplementXGBoost($X_{train}$, $y_{train}$, $X_{val}$, $y_{val}$)}{
    \tcc{1. Initial XGBoost Configuration}
    base\_params $\gets$ \{
      'objective': 'reg:squarederror',
      'eval\_metric': ['rmse', 'mae'],
      'tree\_method': 'hist',  // Faster for larger datasets
      'enable\_categorical': True,  // Native categorical support
      'random\_state': 42
    \}\;
    
    \tcc{2. Hyperparameter Search Space}
    param\_grid $\gets$ \{
      'n\_estimators': [100, 200, 300, 500],
      'max\_depth': [3, 4, 5, 6, 7],
      'learning\_rate': [0.01, 0.05, 0.1, 0.15],
      'subsample': [0.6, 0.7, 0.8, 0.9],
      'colsample\_bytree': [0.6, 0.7, 0.8, 0.9],
      'reg\_alpha': [0, 0.01, 0.1, 1],
      'reg\_lambda': [0.1, 1, 2, 5],
      'min\_child\_weight': [1, 3, 5, 7]
    \}\;
    
    \tcc{3. Randomized Search with Cross-Validation}
    xgb\_model $\gets$ XGBRegressor(**base\_params)\;
    
    random\_search $\gets$ RandomizedSearchCV(
      estimator=xgb\_model,
      param\_distributions=param\_grid,
      n\_iter=100,  // Number of parameter combinations
      cv=5,  // 5-fold cross-validation
      scoring='neg\_mean\_squared\_error',
      n\_jobs=-1,
      verbose=1,
      random\_state=42
    )\;
    
    \tcc{4. Fit with Early Stopping}
    eval\_set $\gets$ [($X_{train}$, $y_{train}$), ($X_{val}$, $y_{val}$)]\;
    
    random\_search.fit(
      $X_{train}$, $y_{train}$,
      eval\_set=eval\_set,
      early\_stopping\_rounds=20,
      verbose=False
    )\;
    
    \tcc{5. Extract Best Model and Parameters}
    best\_model $\gets$ random\_search.best\_estimator\_\;
    best\_params $\gets$ random\_search.best\_params\_\;
    
    \Return best\_model, best\_params\;
  }
\end{algorithm}

\subsection{Feature Importance Analysis}
Native XGBoost feature importance untuk initial understanding:

\begin{algorithm}[H]
  \caption{XGBoost Feature Importance Extraction}
  \label{alg:feature_importance}
  \DontPrintSemicolon
  \SetAlgoLined
  \SetKwProg{Fn}{Function}{:}{}
  \Fn{AnalyzeFeatureImportance($xgb\_model$, $feature\_names$)}{
    \tcc{Get multiple importance types}
    importance\_types $\gets$ ['weight', 'gain', 'cover']\;
    importance\_dict $\gets$ \{\}\;
    
    \ForEach{imp\_type in importance\_types}{
      importance $\gets$ xgb\_model.get\_booster().get\_score(importance\_type=imp\_type)\;
      importance\_dict[imp\_type] $\gets$ importance\;
    }
    
    \tcc{Create importance dataframe}
    feature\_imp\_df $\gets$ pd.DataFrame(importance\_dict)\;
    feature\_imp\_df['feature'] $\gets$ feature\_names\;
    feature\_imp\_df $\gets$ feature\_imp\_df.sort\_values('gain', ascending=False)\;
    
    \tcc{Visualize importance}
    plot\_importance(xgb\_model, importance\_type='gain', max\_num\_features=10)\;
    
    \Return feature\_imp\_df\;
  }
\end{algorithm}

\section{Integrasi Explainable AI}

\subsection{SHAP Implementation untuk XGBoost}
TreeSHAP provides exact Shapley values untuk XGBoost:

\begin{algorithm}[H]
  \caption{SHAP Integration dengan XGBoost}
  \label{alg:shap_xgboost}
  \DontPrintSemicolon
  \SetAlgoLined
  \SetKwProg{Fn}{Function}{:}{}
  \Fn{ImplementSHAP($xgb\_model$, $X$, $feature\_names$)}{
    \tcc{1. Initialize TreeSHAP Explainer}
    explainer $\gets$ shap.TreeExplainer(
      xgb\_model,
      feature\_perturbation='tree\_path\_dependent'
    )\;
    
    \tcc{2. Calculate SHAP Values}
    shap\_values $\gets$ explainer.shap\_values($X$)\;
    expected\_value $\gets$ explainer.expected\_value\;
    
    \tcc{3. Global Feature Importance}
    global\_importance $\gets$ np.abs(shap\_values).mean(axis=0)\;
    importance\_df $\gets$ pd.DataFrame(\{
      'feature': feature\_names,
      'importance': global\_importance
    \}).sort\_values('importance', ascending=False)\;
    
    \tcc{4. Generate Visualizations}
    \tcc{Summary plot untuk global understanding}
    shap.summary\_plot(shap\_values, $X$, feature\_names=feature\_names)\;
    
    \tcc{Dependence plots untuk top features}
    top\_features $\gets$ importance\_df['feature'].head(4)\;
    \ForEach{feature in top\_features}{
      shap.dependence\_plot(feature, shap\_values, $X$, feature\_names=feature\_names)\;
    }
    
    \tcc{5. Individual Explanations}
    \ForEach{idx in sample\_indices}{
      \tcc{Waterfall plot untuk individual prediction}
      shap.waterfall\_plot(shap.Explanation(
        values=shap\_values[idx],
        base\_values=expected\_value,
        data=$X$.iloc[idx],
        feature\_names=feature\_names
      ))\;
    }
    
    \Return shap\_values, expected\_value, importance\_df\;
  }
\end{algorithm}

\subsection{LIME Implementation untuk Patient-Facing Explanations}
LIME untuk quick, intuitive explanations:

\begin{algorithm}[H]
  \caption{LIME Implementation untuk XGBoost}
  \label{alg:lime_xgboost}
  \DontPrintSemicolon
  \SetAlgoLined
  \SetKwProg{Fn}{Function}{:}{}
  \Fn{ImplementLIME($xgb\_model$, $X_{train}$, $X_{test}$, $feature\_names$)}{
    \tcc{1. Initialize LIME Explainer}
    explainer $\gets$ lime.lime\_tabular.LimeTabularExplainer(
      training\_data=$X_{train}$.values,
      feature\_names=feature\_names,
      mode='regression',
      discretize\_continuous=True  // Better untuk patient understanding
    )\;
    
    \tcc{2. Generate Explanations untuk Test Samples}
    lime\_explanations $\gets$ []\;
    
    \ForEach{idx in range(len($X_{test}$))}{
      \tcc{Explain individual instance}
      exp $\gets$ explainer.explain\_instance(
        $X_{test}$.iloc[idx].values,
        xgb\_model.predict,
        num\_features=6,  // Top 6 features
        num\_samples=5000  // Sampling untuk local approximation
      )\;
      
      \tcc{Extract explanation data}
      exp\_dict $\gets$ \{
        'prediction': xgb\_model.predict([$X_{test}$.iloc[idx]])[0],
        'explanation': exp.as\_list(),
        'local\_pred': exp.local\_pred[0],
        'score': exp.score
      \}\;
      
      lime\_explanations.append(exp\_dict)\;
    }
    
    \tcc{3. Generate Visualizations}
    \ForEach{exp in lime\_explanations[:5]}{  // First 5 samples
      exp.as\_pyplot\_figure()\;
    }
    
    \Return lime\_explanations\;
  }
\end{algorithm}

\subsection{Comparative Analysis: SHAP vs LIME}
Systematic comparison untuk optimal usage:

\begin{table}[h]
  \centering
  \caption{SHAP vs LIME Comparison untuk XGBoost Explanations}
  \label{tab:shap_lime_comparison}
  \begin{tabular}{|l|l|l|}
    \hline
    \textbf{Aspect} & \textbf{SHAP} & \textbf{LIME} \\
    \hline
    Computation Time & O(TLD²) - Slower & O(N) - Faster \\
    \hline
    Accuracy & Exact Shapley values & Local approximation \\
    \hline
    Global Insights & Excellent & Limited \\
    \hline
    Patient Understanding & Technical & Intuitive \\
    \hline
    Best Use Case & Regulatory/Clinical & Patient Interface \\
    \hline
  \end{tabular}
\end{table}

\section{Patient-Centric Framework Development}

\subsection{Design Principles}
Framework dirancang dengan prinsip patient empowerment:
\begin{enumerate}
    \item \textbf{Clarity}: Penjelasan dalam bahasa non-technical
    \item \textbf{Interactivity}: User dapat explore different scenarios
    \item \textbf{Actionability}: Insights mengarah pada concrete actions
    \item \textbf{Personalization}: Tailored untuk individual circumstances
\end{enumerate}

\subsection{Dashboard Architecture}

\begin{algorithm}[H]
  \caption{Patient-Centric Dashboard Implementation}
  \label{alg:dashboard}
  \DontPrintSemicolon
  \SetAlgoLined
  \SetKwFunction{Fn}{Function}
  \Fn{BuildPatientDashboard(xgb\_model, shap\_explainer, lime\_explainer)}{
    \tcc{1. Initialize Dashboard Components}
    dashboard $\gets$ \{
      'prediction\_module': PredictionEngine($xgb\_model$),
      'shap\_module': SHAPVisualizer($shap\_explainer$),
      'lime\_module': LIMEInterface($lime\_explainer$),
      'whatif\_module': WhatIfAnalyzer($xgb\_model$),
      'narrative\_module': NarrativeGenerator()
    \}\;
    
    \tcc{2. Prediction Module}
    \Fn{PredictCost($patient\_data$)}{
      prediction $\gets$ xgb\_model.predict($patient\_data$)\;
      confidence\_interval $\gets$ calculate\_prediction\_interval(prediction)\;
      \Return prediction, confidence\_interval\;
    }
    
    \tcc{3. Explanation Module}
    \Fn{GenerateExplanation($patient\_data$, $method$='hybrid')}{
      \If{method == 'detailed'}{
        explanation $\gets$ shap\_explainer.explain($patient\_data$)\;
      }
      \ElseIf{method == 'quick'}{
        explanation $\gets$ lime\_explainer.explain($patient\_data$)\;
      }
      \Else{  // Hybrid approach
        shap\_exp $\gets$ shap\_explainer.explain($patient\_data$)\;
        lime\_exp $\gets$ lime\_explainer.explain($patient\_data$)\;
        explanation $\gets$ combine\_explanations(shap\_exp, lime\_exp)\;
      }
      \Return explanation\;
    }
    
    \tcc{4. What-If Analysis}
    \Fn{WhatIfScenario($patient\_data$, $changes$)}{
      scenarios $\gets$ []\;
      \ForEach{change in changes}{
        modified\_data $\gets$ apply\_change($patient\_data$, change)\;
        new\_prediction $\gets$ xgb\_model.predict(modified\_data)\;
        impact $\gets$ new\_prediction - original\_prediction\;
        scenarios.append(\{change, new\_prediction, impact\})\;
      }
      \Return scenarios\;
    }
    
    \tcc{5. Narrative Generation}
    \Fn{GenerateNarrative($prediction$, $explanation$, $patient\_data$)}{
      narrative $\gets$ []\;
      narrative.append(f"Estimasi biaya asuransi Anda: \${prediction:.2f}")\;
      
      \tcc{Top factors affecting cost}
      top\_factors $\gets$ get\_top\_factors($explanation$, n=3)\;
      \ForEach{factor in top\_factors}{
        impact\_text $\gets$ describe\_impact(factor)\;
        narrative.append(impact\_text)\;
      }
      
      \tcc{Actionable recommendations}
      recommendations $\gets$ generate\_recommendations(top\_factors, $patient\_data$)\;
      narrative.extend(recommendations)\;
      
      \Return " ".join(narrative)\;
    }
    
    \Return dashboard\;
  }
  
\end{algorithm}

\subsection{Interactive Visualizations}
Visualizations designed untuk patient understanding:
\begin{enumerate}
    \item \textbf{Cost Breakdown Pie Chart}: Shows percentage contribution of each factor
    \item \textbf{Feature Impact Bar Chart}: Positive/negative impacts on cost
    \item \textbf{What-If Sliders}: Interactive exploration of scenarios
    \item \textbf{Peer Comparison}: Anonymous comparison dengan similar demographics
    \item \textbf{Trend Projections}: Future cost estimates based on age progression
\end{enumerate}

\section{Evaluasi Sistem}

\subsection{Performance Metrics}
Evaluasi komprehensif XGBoost performance:

\begin{table}[h]
  \centering
  \caption{Evaluation Metrics untuk XGBoost Performance}
  \label{tab:evaluation_metrics}
  \begin{tabular}{|l|l|l|}
    \hline
    \textbf{Metric} & \textbf{Formula} & \textbf{Target} \\
    \hline
    R² Score & $1 - \frac{\sum(y_i - \hat{y}_i)^2}{\sum(y_i - \bar{y})^2}$ & > 0.85 \\
    \hline
    RMSE & $\sqrt{\frac{1}{n}\sum(y_i - \hat{y}_i)^2}$ & Minimize \\
    \hline
    MAE & $\frac{1}{n}\sum|y_i - \hat{y}_i|$ & Minimize \\
    \hline
    MAPE & $\frac{100}{n}\sum|\frac{y_i - \hat{y}_i}{y_i}|$ & < 15\% \\
    \hline
  \end{tabular}
\end{table}

\subsection{XAI Effectiveness Evaluation}
Metrics untuk evaluating explanation quality:
\begin{itemize}
    \item \textbf{Consistency}: Agreement antara SHAP dan LIME rankings
    \item \textbf{Stability}: Variation in explanations dengan different samples
    \item \textbf{Comprehensibility}: User understanding scores (simulated)
    \item \textbf{Computational Efficiency}: Time untuk generate explanations
\end{itemize}

\subsection{System Usability Testing}
Framework evaluation dari patient perspective:
\begin{enumerate}
    \item Response time untuk predictions
    \item Clarity of explanations
    \item Usefulness of what-if scenarios
    \item Overall user satisfaction (simulated metrics)
\end{enumerate}

\section{Ethical Considerations}

\subsection{Data Privacy}
\begin{itemize}
    \item Dataset adalah publicly available dan anonymized
    \item Tidak ada informasi pribadi yang dapat diidentifikasi (PII)
    \item Compliance dengan research ethics guidelines
\end{itemize}

\subsection{Model Fairness}
\begin{itemize}
    \item Analysis untuk demographic bias dalam predictions
    \item Fair representation across regions dan demographics
    \item Transparent reporting of model limitations
\end{itemize}

\subsection{Patient Autonomy}
\begin{itemize}
    \item Predictions presented sebagai estimates dengan confidence intervals
    \item Clear disclaimers tentang model limitations
    \item Emphasis pada informed decision-making, bukan prescriptive advice
\end{itemize}
%

%%%%%%%%%%%%%%%%%%%%%%%%%%%%%%%%%%%%%%%%%%%%%%%%%%%%%%%%%%%%%%%
% Berikut adalah untuk membuat Daftar pustaka
% Style daftar pustaka dapat disesuaikan dengan mengubah 
%\bibliographystyle{kode}
% dengan kode = acm maka hasinya contoh [1]
% dengan kode = agsm maka hasilnya Harvard style (Sukimin, 2017)
%%%%%%%%%%%%%%%%%%%%%%%%%%%%%%%%%%%%%%%%%%%%%%%%%%%%%%%%%%%%%%%%
\cleardoublepage
\addcontentsline{toc}{chapter}{Daftar Pustaka}

\bibliographystyle{plain} %harvard style
\bibliography{References}
%
%\pagebreak
%%%%%%%%%%%%%%%%%%%%%%%%%%%%%%%%%%%%%%%%%%%%%%%%%%%%%%%%%%%%%%%
% Berikut untuk lampiran
%%%%%%%%%%%%%%%%%%%%%%%%%%%%%%%%%%%%%%%%%%%%%%%%%%%%%%%%%%%%%%
\cleardoublepage
\addcontentsline{toc}{chapter}{Lampiran}
\include{Lampiran}
\end{document}
